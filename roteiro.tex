\documentclass[12pt,a4paper]{article}
\usepackage[left=1cm, right=1cm, top=1cm, bottom=2cm]{geometry}
\usepackage[utf8]{inputenc}
\usepackage{enumitem}
\usepackage{multicol, amsmath}
\usepackage{amsfonts}
\usepackage{amssymb}
\usepackage[T1]{fontenc}
\usepackage{graphicx} % Usado para outros tipos de imagens
\usepackage{float} % Usado para posicionamento de imagens
\usepackage{svg}  % Eis o pacote que queremos.
\usepackage{physics}
\usepackage{siunitx}
\usepackage{datetime}
\usepackage{tikz}
\usepackage{href-ul}
\input{commands.tex}
\usetikzlibrary{math}

\begin{document}

\begin{tikzpicture}
    \node at (10.5,4.5) {\Large \textbf{Universidade Federal Rural do Semi-Árido}};

    \node (u) at (1.5,2.5) {\includegraphics[height=5 cm, width=3 cm]{BrasaoUfersa.png}};
    \node at (10.5,3.5) {\Large Departamento de Engenharias e Tecnologia};

    \draw (3.5,2.75) -- (17.5,2.75);

    \node at (6.75,2) {LAED II};
    \node at (6.75,1) {Professor: Kennedy Lopes};

    \node at (14.25, 2) {PET2037 e PEX1247} ;
    \node at (14.25, 1) {Data: \ddmmyyyydate\today} ;

    \node at (9, -1) {\LARGE \underline{\textbf{Roteiro 03}}};
\end{tikzpicture}
\thispagestyle{empty}

\large
\section*{Introdução}

A lista de cidades apresentada no arquivo {\color{blue}\href{https://github.com/kennedyufersa/hashTable/blob/main/bancoDeDados/coordenadas.csv}{cidades.csv}} se encontram geograficamente nas localizações indicadas no arquivo {\color{blue}\href{https://github.com/kennedyufersa/hashTable/blob/main/bancoDeDados/coordenadas.csv}{coordenadas.csv}}. 

Sabendo disso, pode-se calcular com esses arquivos as distâncias entre as cidades A e B, a partir de suas latitudes e longitudes:

$$dist(A,B) = \sqrt{(A.x - B.x)^2 + (A.y - B.y)^2}$$

Sendo $A.x$ e $A.y$ a longitude e latitude da cidade A e $B.x$ e $B.y$ são a latitude e longitude da cidade B.

Exemplo hipotético:
\begin{itemize}
    \item Cidade A(Lat 10; Log 20);
    \item Cidade B(Lat 15; Log 18);
\end{itemize}

\begin{align}
    dist(A,B) &= \sqrt{(A.x - B.x)^2 + (A.y - B.y)^2}\\
    dist(A,B) &= \sqrt{(10 - 15)^2 + (20 - 18)^2}\\
    dist(A,B) &= \sqrt{(-5)^2 + (-3)^2}\\
    dist(A,B) &= \sqrt{25 + 9}\\
    dist(A,B) &\approx 5.83\ km
\end{align}

Considere que as distâncias são medidas em $km$.

\section*{Novo cálculo de vizinhança}

A vizinhança entre cidades pode ser definida pelo motivo das cidades terem uma fronteira em comum, ou não. Deste modo, a cidade de Pau dos ferros - por exemplo - tem as cidades São Francisco do Oeste, Francisco Dantas, Serrinha dos pintos, Antônio Martins, Rafael Fernandes e Encanto como vizinhos.

Mas apesar disto, outras cidades são influenciadas ou influenciam a cidade de Pau dos Ferros (como Portalegre, Martins Alexandria, São Miguel, entre outras) mesmo não sendo vizinhos.

Sabendo disso, a proposta desse roteiro é definir uma nova medida de vizinhança baseado na distância entre as cidades. Nosso princípio será o seguinte: Duas cidades serão consideradas vizinhas (ou influentes) se estiverem a uma distância mínima $D$ entre elas.

Um exemplo (diferente do nosso, mas não tão diferente) de influência entre as cidades podem ser visto na figura \ref{fig:fig1}. Nesta figura a influência das cidades são baseadas pela características socioeconômicas que influenciam cada região.

\begin{figure}[H]
    \centering
    \includegraphics[width=\linewidth]{altoOeste.jpeg}
    \caption{Microrregiões do RN Fonte: \href{https://pt.wikipedia.org/wiki/lista_de_mesorregi\%C3\%B5es_e_microrregi\%C3\%B5es_do_Rio_Grande_do_Norte}{microrregiões do RN}} 
    \label{fig:fig1}
\end{figure}

\newpage

\section*{Exercício avaliativo:}

Construa um Grafo na qual:

\begin{itemize}
    \item Os \textbf{Vértices} são as cidades.
    \item Se duas cidades estão a uma distância de até $D\ km$, então existe uma \textbf{Aresta} que as conectam.
\end{itemize}

\subsection*{Questão Única}
\begin{enumerate}[label=\Roman*)]
    \item Calcule a distância entre cada dois pares de cidades (vértices). Se a distância for menor do que uma distância $D\ km$\footnote{A distância D é um parâmetro da função que o grafo será construído}, conecte as cidades por uma aresta ponderada (o peso entre das arestas é igual a distância entre elas).
    \item Identifique a cidade (vértice) que tem mais cidades vizinhas (grau dos vértices) no Rio Grande do Norte em função da distância D.
    \item Qual cidade não tem vizinhos, baseados na distância D?
    \end{enumerate}

    As respostas devem estar no arquivo main.cpp com o seguinte modelo apresentado \href{https://github.com/kennedyufersa/scriptLab/main.cpp}{neste arquivo}.
\end{document}